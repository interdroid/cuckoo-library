\documentclass{article}

\usepackage{graphicx}
\usepackage{epstopdf}
\usepackage{listings}
\usepackage{multirow}
%\usepackage{pst-tree}
\usepackage{setspace}
\usepackage{subfig}
\usepackage{url}
\usepackage{verbatim}
\usepackage{array}

\usepackage{hyperref}
\usepackage{color}
% \usepackage{textcomp}
\usepackage{rotating}
\usepackage{tabularx,colortbl}
\definecolor{listinggray}{gray}{0.9}
\definecolor{lbcolor}{rgb}{0.9,0.9,0.9}
\lstset{
	backgroundcolor=\color[rgb]{0.95,0.95,0.95},
	tabsize=4,
	rulecolor=,
	language=java,
    basicstyle=\scriptsize,
    columns=fixed,
    showstringspaces=false,
    extendedchars=true,
    breaklines=true,
    prebreak = \raisebox{0ex}[0ex][0ex]{\ensuremath{\hookleftarrow}},
    frame=single,
    showtabs=false,
    showspaces=false,
    showstringspaces=false,
    identifierstyle=\ttfamily,
    keywordstyle=\color[rgb]{0,0,0.5},
    commentstyle=\color[rgb]{0.133,0.545,0.133},
    stringstyle=\color[rgb]{0.627,0.126,0.941},
}

\begin{document}

\title{Cuckoo Manual}

\maketitle


\section{Introduction}
Cuckoo is a framework for computation offloading applications on Android. Some
applications might contain compute intensive tasks that can better be executed
on a \emph{remote} resource, such as a laptop, desktop or cloud environment,
than on the device itself. For such tasks computation offloading can be used.
Computation offloading transfers execution of compute tasks to remote resources
to get the result of the task quicker or to save energy usage on the mobile
device.

The Cuckoo framework aids application developers in transforming a regular
Android application into a computation offloading application. Furthermore, at
runtime it decides whether or not to offload the particular task. This decision
has to be made, because the circumstances (network type and status, invocation
paramaters of the method call) of method invocations on mobile devices change
continuously, making offloading sometimes beneficial and sometimes not.

Furthermore, the Cuckoo framework contains a server application that can be
installed on remote resources and a resource manager app for mobile phones, that
can manage the remote resources that are known to the mobile device and can be
used for apps on that mobile device that are built with Cuckoo.

This document describes how to use Cuckoo. It describes how to write a Cuckoo
app and how to run this app. For more in depth discussion of the framework,
please read Chapter 2 of the PhD thesis: Programming Frameworks for Distributed
Smartphone Computing by Roelof Kemp.

\section{Write a Cuckoo App}
In this section we provide a step by step guide on how to write a Cuckoo
application.

\subsection{Prerequisites}
\begin{itemize}
  \item Eclipse integrates with the Android plugin for the Eclipse IDE. Download
  and install Eclipse and the Android plugin (ADT) as described at:
  \url{http://www.eclipse.org} and
  \url{http://developer.android.com/tools/sdk/eclipse-adt.html}. Please check
  the versions of the Android Plugin and SDK with the SDK Manager in Eclipse.
  The Android SDK Build Tools are tested with version 18.0.1, the Android SDK
  Tools with version 22.0.5.
  \item Download the Cuckoo library from github: \\
  \url{https://github.com/interdroid/cuckoo-library}.\\
  Compile the library to
  generate the appropriate jar files using ant in the project root: \\
  \verb!$ ant!\\
  \item Install the Cuckoo plugin for Eclipse. Open Eclipse. Click the ``Help''
  menu, select ``Install New Software\ldots''. Enter the following in the ``Work
  with'' field:
  \url{https://raw.github.com/interdroid/cuckoo-plugin-feature/master/exported}, press Enter.
  Make sure that you uncheck: ``Group items by category''. Select ``Cuckoo
  Plugin Feature''. Press the ``Next'' button. Continue the wizard until the
  plugin is installed. Restart Eclipse.
  \item Set the Cuckoo Library Location in the Eclipse Preferences (Mac:
  Eclipse $>$ Preferences, Ubuntu: Window $>$ Preferences).
  Open the Eclipse Preferences from the menu. Select ``Cuckoo Preferences'', enter the
  path to the root directory of the Cuckoo library, make sure that the library
  contains the jar files in the ``lib'' directory.
\end{itemize}

\subsection{Make a project offloadable}
Write the Android application. Make sure that any compute intensive code
is performed in an Android Service (see:
\url{developer.android.com/reference/android/app/Service.html}) and that this
service runs in a separate process. To make a service running in a separate process, you can add the following to the service tag in the AndroidManifest.xml:
android:process=":remote". You should communicate with the service by binding to
it and invoking methods on the proxy. For this reason you should write an AIDL
file that exposes the interface of the service and implement this interface
accordingly in the service (see:
\url{http://developer.android.com/guide/components/aidl.html}).

So far it's all basic Android stuff. Now the only thing you have to do to make
the app offloadable is to add the Cuckoo ``nature''. You can do this by
right-clicking on the project and selecting ``Make Offloadable with Cuckoo''
from the ``Android Tools'' menu item in the context menu.

Then the Cuckoo libraries are added to your project (i.e. in the libs
directory), a dummy remote implementation is generated in the directory
``remote'' within your project root directory, and the jar resulting from this
remote implementation is put in the ``assets'' folder of your project. Furthermore the
code that is generated for the stub/proxy pair as a result of the AIDL
specification is altered, so that Cuckoo calls are inserted.

Now you should replace the dummy remote implementation with a real remote
implementation. You might want to copy-paste the local implementation from your
service, or create an entirely different implementation. Keep in mind that the
remote code cannot access Android specific libraries that are not purely
implemented in Java. The Eclipse plugin automatically compiles your Java code
into the jar that is placed in the assets folder. You can use external libries
for your remote implementation. Place them in the
``remote/$<$packagename$>$/external'' directory.

Make sure you add the following permissions to the AndroidManifest as they are
required by the Cuckoo runtime:

\begin{itemize}
\item  android.permission.INTERNET
\item  android.permission.ACCESS\_WIFI\_STATE
\item  android.permission.ACCESS\_NETWORK\_STATE
\item  android.permission.ACCESS\_FINE\_LOCATION
\end{itemize}

Now the application is ready to be deployed on the device.

\subsubsection{Control over offloading}
\begin{itemize}
  \item In the AIDL file one can describe the offloading strategy.  The strategy can be
``local", ``remote'', ``energy'', ``speed'' and ``parallel''. Cuckoo also
supports a multifold offloading strategy of ``energy'' combined with ``speed''.
In this case Cuckoo will offload the execution of the method if any of the
strategies results in a positive answer on the question:
Should we offload? The order of consulting the model for energy usage and
execution time depends on the order within the purpose description --
``energy/speed'' will first consult the energy model. If the strategy is remote,
Cuckoo will always try to offload the method, because the remote implementation
will for instance deliver a better quality result. By default the purpose of
offloading is ``speed/energy''. You can add a strategy by providing a special
line with comments above the method in the AIDL file as shown in Figure
\ref{figure-aidl}:

\begin{figure}[bt!] 
\centering
\begin{lstlisting}[language=java]
package interdroid.photoshoot;

import interdroid.photoshoot.Face;

// cuckoo.strategy=speed
interface FaceDetectorInterface {
  List<Face> findFaces(in byte[] jpegData, int width, int height, int maxFaces);
}
\end{lstlisting}

\begin{lstlisting}[language=java]
float weight_findFaces(in byte[] jpegData, int width, int height, int maxFaces) {
    return width * height;
}
\end{lstlisting}

\begin{lstlisting}[language=java]
long returnSize_findFaces(in byte[] jpegData, int width, int height, int maxFaces) {
    return 100;
}
\end{lstlisting}
\caption{The interface of the face detection service in AIDL. The services will
search for a maximum number of faces in the provided image and returns a list
of Face objects that it has found. }
\label{figure-aidl} 
\end{figure}

\item You can help Cuckoo in estimating the runtime of a method by overriding
the method weight function in the Stub implementation in the service. This
weight method has the same signature as the method defined in the AIDL file, but
its name is prefixed with \verb!weight_! and the return type is of type double.
The default implementation returns 1, but if the execution time of the method
depends on the variables, you can assign a weight based on the variables. For
instance an image operation that scales per pixel can return the number of
pixels as the method weight. Any measured runtime results will then be
normalized to the time per pixel and subsequently a better estimate can be made
for a new invocation with a known amount of pixels.
\item Likewise, you can help Cuckoo in estimating the return size of the method
by overriding the \verb!returnSize_! variant of the method in the Stub.
\item If you want to disable a particular method for offloading, put this line
in front of it in the AIDL file:\\
\verb!// cuckoo:enabled=false!  
\item Some phones put the WiFi radio in a lower power state after some time,
which can delay receiving the response with up to hundreds of milliseconds. To
prevent this Cuckoo can keep the radio active continuously, or try to activate
the radio just in time. This can be configured through the following system
property.\\
\verb!System.setProperty(Cuckoo.WIFI_WAKE_STRATEGY,! \\
\verb!    Cuckoo.WIFI_WAKE_STRATEGY_SLEEP);!\\
Cuckoo supports the following strategies WIFI\_WAKE\_STRATEGY\_SLEEP (does not
try to keep the radio active, default if the offloading strategy is ``energy''),
WIFI\_WAKE\_STRATEGY\_AWAKE, (default if the offloading strategy is ``speed'') and
WIFI\_WAKE\_STRATEGY\_JIT (which tries to wake the WiFi radio just in time).
\end{itemize}

\section{Run a Cuckoo App}
In this section we describe how to run a Cuckoo App. To run a Cuckoo app a
couple of things have to be done. There must be a remote Cuckoo server running.
This server should be bound to the mobile device. And the mobile device should
contain an app with offloadable code.

\subsection{Start Cuckoo Server}
To start a Cuckoo server, download the code from:
\url{https://github.com/interdroid/cuckoo-library}. Compile the code using ant:
\\
\verb!$ ant!\\
Then start the Cuckoo server using the command:\\
\verb!$ java -cp .:* -Dlog4j.configuration=file:path/to/log4j.properties!
\verb!  interdroid.cuckoo.server.CuckooServer! \\
If you run this command from graphic user environment, a popup with a QR-code is
shown, otherwise the server generates a file 'qr.png' in the root directory. The
QR-code is needed to bind the server to the mobile device.

\subsubsection{Advanced Items for the Cuckoo Server}
\begin{itemize}
  \item If you use SSH to access your remote machine you can start the server with
'nohup' to keep it running even if your SSH connection terminates.
\item By default the server runs on port 9000 and assumes its IP-address can be
reached by the mobile device (e.g. public or in the same local area network).
Firewalls may prevent connections. 
\item The server produces output to standard err/out and with log4j. Both can
with the appropriate redirecting be stored in different files.
\item The server caches code it receives from mobile devices. If you update your
code on the mobile device you should delete the code from the 'services'
directory at your server and also restart the server.
\item The Cuckoo server can be configured using a configuration file called
``cuckoo.properties'' in the root directory of the project. Such a file looks
like this:\\

\begin{lstlisting}[language=java]
# cuckoo properties

# ipaddress is determined automatically, if this use the property
cuckoo.server.ipaddress = 123.123.123.123
# upload speed to internet in bytes/ms (equiv. to kilobytes/second)
cuckoo.server.upload = 10000
# upload variance
cuckoo.server.upload.variance = 1000
# download speed from internet in bytes/ms (equiv. to kilobytes/second)
cuckoo.server.download = 7500
# download variance
cuckoo.server.download.variance = 1000
# geolocation of server (latitude,longitude)
cuckoo.server.location = 52.333045,4.867841
# bssids of accesspoints within the LAN of the resource (comma separated list)
cuckoo.server.bssids = 0:12:7f:50:a4:10,0:12:7f:50:a4:13

\end{lstlisting}
\end{itemize}

\subsection{Connect Server with Mobile Device}
Cuckoo comes with a separate app, which is called the Resource Manager, which
you can download from
\url{https://github.com/interdroid/cuckoo-resource-manager}.
Use this app to scan the QR code generated by starting the service. By scanning
the QR-code the mobile device knows the remote resource and can use it for
offloading.

\subsection{Run the Cuckoo App}
Now everything is ready for the offloading app to run. The first time it will
run in parallel, both local and remote to get statistics. Also the jars from the
assets directory will be transfered to the remote resource.


\end{document}
